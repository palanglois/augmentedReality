\documentclass[a4paper,10pt]{article}
%\documentclass[a4paper,10pt]{scrartcl}

\usepackage[utf8]{inputenc}
\usepackage{natbib}
\usepackage[margin=0.5in]{geometry}



\title{Project proposal - Object Recognition and Computer vision}
\author{Othman Sbai, Pierre-Alain Langlois}
\date\today

\pdfinfo{%
  /Title    (Project proposal - Computer vision - Sbai & Langlois)
  /Author   (Othman Sbai, Pierre-Alain Langlois)
  /Creator  (Pierre-Alain Langlois)
  /Producer (Othman Sbai)
  /Subject  (Augmented reality - tracking)
  /Keywords (augmented reality neural network ponts)
}

\begin{document}
\maketitle

\section{Chosen topic}
We have chosen the topic : A.3 Instructions for assembling simple lego objects.

\medbreak
The advent of augmented reality tools such as Microsoft Hololens have made possible many interesting 
applications that augment the visual experience of the user providing relevant informations and 
distractions. Across the web, one can find lots of tutorial videos for performing a certain task be it 
assemble a furniture, prepare a meal, change tire in addition to DIY videos. These explanations can 
really be enhanced through augmented reality technology by overlaying instructions in the view of Hololens 
for example in order to adapt the tutorial to the real world’s configuration. 
\medbreak

We propose to tackle the problem of providing instructions for assembling simple LEGO set. This simple 
game-related problem is a good start in manipulating and recognizing 3D objects from a head-mounted camera. 
Our goal is to create a assistant that is able to recognize the state of the LEGO mounting problem and 
suggest the next step by blending virtual movement on the reality perceived by the player thanks to 
Hololens.

\medbreak
Some researchers already worked on the problem of LEGO brick identification and retrieval in realtime from 
2D images\cite{botha_realtime_2009}.
 
\section{Plan of work}

\begin{itemize}
 \item GOAL: Implement an interactive assistant that helps solving/assembling a LEGO set.
 \item We assume the knowledge of the set of sequences required for the assembly, this can be either 
 supplied by LEGO from the manual, or can be deduced from tutorial videos, as was done in the paper 
 \cite{alayrac_unsupervised_2015}.
 \item From a head mounted camera, recognize LEGO part to be moved and its destination and display a 
 hint of the movement overlaid on the hololens.
 \item Recognize similar/equivalent parts to be moved

\end{itemize}


\section{Operational organization}

\subsection{Groupe members}

\begin{itemize}
 \item Othman Sbai
 \item Pierre-Alain Langlois
\end{itemize}


\subsection{Plans for work sharing}

\bibliographystyle{plain}
\bibliography{biblio}


\end{document}
